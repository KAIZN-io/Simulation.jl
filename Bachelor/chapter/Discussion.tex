\section{Discussion}
Distinctions between the peaks of the membrane potential in \ref{IonPaper} seems to have its root in the different time grid setting of the ODE-Solver in Copasi, which was used for the ion model simulation in the paper \cite{Gerber_2016}.\\
Martina Fröhlich (one of the author of the ion model) noted in her dissertation, that the ion model should only be used for analyses with starved cells \cite{martinafroehlich}.\\\\
A current limitation of the combined model is that the concentration of $Cl_{in}$ high and that the $Glyc_{in}$ grows limitless because the cell can not restore its full turgor pressure and the expression for Hog1PPn do not turn off.\\ The solution should be a better parametrisation or a better inital value set for the ODEs. The used setting for the combined model is the best result of multiple manual attempts to get the simulated system near to a steady state in the abscence of any stimulus..\\\\
It is assumed that there are not any temperature gradient ($grad(T)=0$) which would results in a heat flux. Only under well controled experimental conditions this assumption holds true.\\\\
One of the problems in merging kinetic models from different sources and experimental conditions is the parameterization process of the merged model \cite{Wang2017}. With no data available it was a try and error process to find a good parameter value for a substance, which is in a interplay with many variables. Available data sets could help in this process but even with this the parameterizations task remains a challenge. \cite{Ke_2013}. \\\\
The  \ref{hogImplemented} for the response of $Hog1PPn$ for a single NaCl impulse for the hog model is not the same as in the paper. Nevertheless, the simulated behaviour reproduces experimental data for this stimulus. It is suspected, that not the final version of the model were visualized for the paper.\\\\
\newpage

\subsection{Outlook}
Even tough I am aware of that a full understanding of complex nonlinear systems might not be ever possible \cite{Noell2018}, 
I would like to expand the work with the model to the point where I am able to control plasma membrane transport mechanismen in eukaryotic single cells. I propose, that a precise analyse of the extracellular environment could allow me then to get an idea what happend inside the cell to get this meassured secretion. For this, secretory pathways (SEC) must be implemented to expand the stress-response network for the desired approach.\\
Another further work aspect is the improvement of the model vision control with the help of a database. \\
A general problem with a personal designed programming environment is that it quick needs multiple programms to be installed on the server and several configuration to be made. This way it is not easy to run a simulation on another computer system. To address this problem, the software Docker will be implemented in the near future. With Docker you allow other users to run your program and its dependencies. \\\\I propose that this would allow us to perform an efficient way for a structural analysis of complex mathematical systems, because it would save resources if you can simulate easily from every computer system in the same environment even at the same time. This is called computer cluster and could enhances us to find an optimal configuration for a model with the help of machine learning. This should be my main task for the future with the focuse on DoE (Design of Experiment).
