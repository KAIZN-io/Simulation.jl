\documentclass[12pt,a4paper,bibliography=totocnumbered]{scrartcl}
\usepackage[utf8]{inputenc}
\usepackage[ngerman]{babel}
\usepackage[T1]{fontenc}
\usepackage{amsmath,amssymb,amstext}
\usepackage{graphicx}
\usepackage{subcaption}
\usepackage[font=small,labelfont=bf]{caption}
\usepackage{fancyhdr}
\usepackage{titling}
\usepackage{color}
\usepackage{array}
\usepackage{chemmacros}
\usepackage{chemexec}
\usepackage{chemfig,chemnum}
%\usepackage[version=4]{mhchem}
\usepackage{float}
\usepackage{natbib} 
\usepackage[hyphens,spaces]{url}
\newcolumntype{M}[1]{>{\centering\arraybackslash}m{#1}}

\begin{document}
	
\section{bachelor thesis }
Test

\section{system biology}
TEst2

\section{bachelor thesis theory}
Osmolarity means the amount of substance $n$ of osmotic activ particles per volume $V$ of the solution. \\
Osmotic pressure happens if two solutions with different osmolarity are seperated by a semipermeable membrane, where not all substances can pass the membrane for creating an equilibrium.\\ 
S. cerevisiae has a plasma membrane which functions as a semipermeable membrane. Water can diffuse freely through it, to adapt to the osmotic changes. At a hyperosmotic shock in the extracellular environment, water will flow out of the cell and let the cell shrink in volume. Cause of the shock event the high osmolarity glycerol (HOG) pathway in S. cerevisiae gets activ and synthesis the osmolyte glycerol. With that the cell increases the internal osmolarity and restores it's volume cause of the influx of water. \\
Additionally to the activation of the HOG pathway, the glycerol channel Fps1 in the membran closes. With a closed channel, glycerol can't leak out of the cell. This supports the accumulation of glycerol inside the cell. After the adaption to the new extracellular environment Fps1 opens.
\section{useful unit conversion}
\begin{equation*}
	\frac{mol}{m^3}=mM
\end{equation*}
\begin{equation*}
\sqrt[3]{fL}=\mu m
\end{equation*}

\section{math and physics}
\begin{equation*}
div(a\vec{u}) = div(\vec{u})a + grad(a)\vec{u}
\end{equation*}

\section{theoretical biophysics}
\begin{equation*}
\frac{d\vec{x}}{dt} = \vec{f}(\vec{x},\vec{p},t)
\end{equation*}
Jacobi-matrix $A$ characterize the steady state.
\begin{equation*}
A = \{a_{ij} \}  =  
\begin{pmatrix}
\frac{\partial f_1 }{\partial x_1}& \frac{\partial f_1 }{\partial x_2} & ... \\
\frac{\partial f_2 }{\partial x_1} & \frac{\partial f_2 }{\partial x_2} & ... \\
... & ... & ...
\end{pmatrix}
\end{equation*}
Eigenvalue equation: $(A-I\lambda)\vec{b}=0$ , with $\vec{b}$ : Eigenvector and $\lambda$ : Eigenvalues\\
Eigenvalue problem: $Det(A-I\lambda)= \vec{0}$\\
the steady state is unstable, if the Realteil of one Eigenvalue is positiv! \\\\
\underline{Example}:  
\begin{equation*}
\underset{v_0}{\rightarrow} S_1 \underset{v_1}{\rightarrow} S_2 \underset{v_2}{\rightarrow} 
\end{equation*}
We assume that the reactions are linear
\begin{equation*}
\frac{dS_1}{dt} = v_0 - k_1S
\end{equation*}
\begin{equation*}
\frac{dS_2}{dt} = k_1S -k_2S
\end{equation*}
written in a vector / matrix form
\begin{equation*}
\frac{d}{dt} \begin{pmatrix} S_1 \\ S_2 \end{pmatrix} = 
\begin{pmatrix} -k_1 & 0 \\ k_1 & -k_2 \end{pmatrix}  \begin{pmatrix} S_1 \\ S_2 \end{pmatrix} 
+  \begin{pmatrix} v_0 \\ 0 \end{pmatrix}
\end{equation*}
with 
\begin{equation*}
	A  = \begin{pmatrix} -k_1 & 0 \\ k_1 & -k_2 \end{pmatrix} 
\end{equation*}


\section{thermodynamic}
Free energy is the part of the energie, with which we really can do work $A$.\\

Every time a force $X_j$ causes a fllux $J_i$, than there is also a force $X_i$ that causes a flux $J_j$\\\\
A flux $J_k$ over the membran is calculated with the Nernst-Planck equation:
\begin{equation*}
J_k = -u_kRT\frac{dc_k}{dt} -u_kz_kFc_k\frac{d\phi}{dx}
\end{equation*}
If a process has a constant pressure $p$ and a constant temperatur $T$, then we calculate with the free enthalpie $G(T,p)$.\\\\
diffusion of uncharged substances: calculated with the 2. Fick's law of diffusion:
\begin{equation*}
\frac{dc_k}{dt}=u_kRT \Delta c_k
\end{equation*}

\subsection{irreversible thermodynamic}
\underline{Def.}.: Analysis of spartial inhomogeneous systems and of time dependent processes.\\\\
Forces caused by ATP hydrolysis depend on the reaction affinity $A_{Ar}$
\begin{equation*}
A_{Ar} = \frac{RT}{\bar{c}_{ATP}}(c_{ATP}-\bar{c}_{ATP})(1+K_{eq})
\end{equation*}
The equation for the flux of ion k is:
\begin{equation*}
J_k = \sum_{j=1}^n L_{kj}(RT\cdot ln\left(\frac{c_k^{in}}{c_k^{out}}\right) + z_kF\Delta \phi ) + L_{kAr}A_{Ar}
\end{equation*}
other ways to trigger fluxes:
grad(T) --> $\vec{J}_Q$ / $\vec{J}_{elek}$ / $\vec{J}_{Diff}$\\
For a contact of two hydrophil surfaces (e.g. biological membranes) the water must be removed. The forces needed for that is called hydration power.

\section{experimental techniques}
Auswahlregel IR : Dipolmoment muss sich ändern oder entstehen.\\
Anwendung IR : Strukturaufklärung organischer Moleküle; Proteinanalytik mit FT-IR Spektroskopie; Sekundärstrukturbestimmung von Proteinen \\
CD Spektroskopie (im UV/vis-Bereich) : Nur bei chiralen Molekülen möglich \\\\
\underline{Spektrometer}:\\
Lichtquelle --> Drehbares Gitter  (oder Fouriertransformation (FT))--> Blende --> Strahlenteiler --> Probe bzw. Referenz --> Monochromator --> Detektor\\\\
\underline{Evaneszenz}:\\
beschreibt das Phänomen, dass Wellen in ein Material, in dem sie sich nicht ausbreiten können, eindringen und unter dessen Oberfläche expontentiell abklingen.\\\\
$H^+$-Transport über Membran (in die Zelle hinein) erhöht den elek. chem. Gradienten, welcher als Energiequelle für ATP-Produktion (über die F0F1-ATPase) und andere Transporte dient. Die Na-ATP-Synthase wird durch $\Delta Na$ getrieben und verfolgt nach dem gleichen Prinzip.\\ Das auch der elek. Gradient die ATP-Synthese bei $H^+$ ATP-Synthese ermöglicht, wird über die <<Proton Well Hypothese>> angenommen.\\\\
$Na^+$ -Hydrate > $K^+$ -Hydrate, aber $Na^+<K^+$
\subsection{experimental part of thesis}
ergodic hypothesis : $\langle A \rangle_{ensemble}$ = $\langle A \rangle_{time}$\\
Low volume, high pressures and fast registration times are needed for the stopped-flow-methode (FT-IR methode).

\section{statistic}
Pearson correlation coefficient $r_{xy}$
\begin{equation*}
r_{xy}=\frac{s_{xy}}{s_x \cdot s_y}=\frac{Cov(x,y)}{Var(x)\cdot Var(y)}=Corr(x,y)
\end{equation*}
Note: even if $Corr(x,y)$ has a value, $x$ and $y$ mustn't influence each other. They could both depend on a third variable. In this case $Corr(x,y)$ is a >>Scheinkorrelation<<. Or there can also be no connection between $x$ and $y$ at all.\\\\
displacement law:
\begin{equation*}
	Var(X|Y)=E(X^2|Y)-(E(X|Y))^2
\end{equation*}
decomposition of variance:
\begin{equation*}
Var(X)=E[Var(X|Y)] + Var[E(X|Y)]
\end{equation*}
General decompostion of variance allows the variance to be decomposed into as many components as there are potential sources of variation.

\section{possible content for Ph.d.}
Hyperosmotic stress response in yeast, cause of extracellular salt. Get insight of the behaviour of the cell through analysis of the extracellular change with the time.

\section{all about fluxes over the plasma membrane  in yeast}
hypo osmotic shock --> rapid glycerol efflux to prevent cells from bursting\\
Fps1 mediates bidirectional transport of glycerol across the plasma membrane\\
It's suggested that HOG pathway controls Fps1 regulation for both hyper and hypo osmotic stress

\section{work with data}
measured data --> analytic data --> computational results --> figures/tables --> published article\\
documentation tool for Python: Sphinx

\section{quotation}
>>Wenn zwei ähnlich gespannte Saiten gleichzetig gezupft werden, ergibt sich immer dann ein angenehmer Klang, wenn die Längen der Saiten in einfachem ganzzahligen Verhältnis zueinander stehen.<<
\section{interesting to look after - type in}
>>success rate phase 1 to approval<< - google pictures\\
>>valuation in life sciences a practical guide<<\\
>>business development for the biotechnology and pharmaceutical industry<<\\
ca. $90 \%$ aller $Ca^{2+}$-Ionen in Zellen sind gepuffert! Nimmt die Pufferkapazität mit den Jahren ab?

	\bibliographystyle{siam} 
	\renewcommand{\refname}{Literaturverzeichnis} 
	\bibliography{Literatur}\newpage
\end{document}